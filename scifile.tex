% Use only LaTeX2e, calling the article.cls class and 12-point type.

\documentclass[12pt]{article}

% Users of the {thebibliography} environment or BibTeX should use the
% scicite.sty package, downloadable from *Science* at
% www.sciencemag.org/about/authors/prep/TeX_help/ .
% This package should properly format in-text
% reference calls and reference-list numbers.

\usepackage{scicite}

% Use times if you have the font installed; otherwise, comment out the
% following line.

\usepackage{times}
\usepackage{graphicx}

% The preamble here sets up a lot of new/revised commands and
% environments.  It's annoying, but please do *not* try to strip these
% out into a separate .sty file (which could lead to the loss of some
% information when we convert the file to other formats).  Instead, keep
% them in the preamble of your main LaTeX source file.


% The following parameters seem to provide a reasonable page setup.

\topmargin 0.0cm
\oddsidemargin 0.2cm
\textwidth 16cm 
\textheight 21cm
\footskip 1.0cm


%The next command sets up an environment for the abstract to your paper.

\newenvironment{sciabstract}{%
\begin{quote} \bf}
{\end{quote}}


% If your reference list includes text notes as well as references,
% include the following line; otherwise, comment it out.

\renewcommand\refname{References and Notes}

% The following lines set up an environment for the last note in the
% reference list, which commonly includes acknowledgments of funding,
% help, etc.  It's intended for users of BibTeX or the {thebibliography}
% environment.  Users who are hand-coding their references at the end
% using a list environment such as {enumerate} can simply add another
% item at the end, and it will be numbered automatically.

\newcounter{lastnote}
\newenvironment{scilastnote}{%
\setcounter{lastnote}{\value{enumiv}}%
\addtocounter{lastnote}{+1}%
\begin{list}%
{\arabic{lastnote}.}
{\setlength{\leftmargin}{.22in}}
{\setlength{\labelsep}{.5em}}}
{\end{list}}

%%%%%%%%%%%%% ############# %%%%%%%%%%%
\newcommand{\isro}{{\it ISRO}}
\newcommand{\suit}{{\it{SUIT}}}
\newcommand{\degree}{$^{\circ}$}
\newcommand{\sr}[1]{{\bf\color{red} [#1]}}
%%%%%%%%%%%%% ############# %%%%%%%%%%%


% Include your paper's title here

\title{A simple {\it Science\/} Template} 


% Place the author information here.  Please hand-code the contact
% information and notecalls; do *not* use \footnote commands.  Let the
% author contact information appear immediately below the author names
% as shown.  We would also prefer that you don't change the type-size
% settings shown here.

\author
{John Smith,$^{1\ast}$ Jane Doe,$^{1}$ Joe Scientist$^{2}$\\
\\
\normalsize{$^{1}$Department of Chemistry, University of Wherever,}\\
\normalsize{An Unknown Address, Wherever, ST 00000, USA}\\
\normalsize{$^{2}$Another Unknown Address, Palookaville, ST 99999, USA}\\
\\
\normalsize{$^\ast$To whom correspondence should be addressed; E-mail:  jsmith@wherever.edu.}
}

% Include the date command, but leave its argument blank.

\date{}



%%%%%%%%%%%%%%%%% END OF PREAMBLE %%%%%%%%%%%%%%%%



\begin{document} 

% Double-space the manuscript.

\baselineskip24pt

% Make the title.

\maketitle 



% Place your abstract within the special {sciabstract} environment.

\begin{sciabstract}
  In this article, we examine the first solar flare localized by the flare detection algorithm of Solar Ultraviolet Imaging Telescope (\suit), onboard the Aditya-L1 payload. \suit~observes the Sun in eleven pass bands in the 200-400 nm regime. This provides us with a unique opportunity to observe Solar flares and their effect on the local solar environment in some of the wavelength bands for the first time. We compare the {\it SUIT} observations with observations from other observatories, e.g. {\it SDO}/AIA, {\it IRIS}, {\it SO}/STIX. We observe umbral bright kernels in the red wing of Mg II. We use {\it Chandrayan-2}/XSM observations to conclude that these bright kernels are Photospheric and appear due to heating in the Photosphere by the X-ray emitted from the flaring plasma.
\end{sciabstract}



% In setting up this template for *Science* papers, we've used both
% the \section* command and the \paragraph* command for topical
% divisions.  Which you use will of course depend on the type of paper
% you're writing.  Review Articles tend to have displayed headings, for
% which \section* is more appropriate; Research Articles, when they have
% formal topical divisions at all, tend to signal them with bold text
% that runs into the paragraph, for which \paragraph* is the right
% choice.  Either way, use the asterisk (*) modifier, as shown, to
% suppress numbering.


Over the past few decades, solar flares have been mostly imaged in Extreme Ultraviolet with Atmospheric Imaging Assembly onboard Solar Dynamic Observatory ({\it SDO}/AIA, \cite{sdo,aia}), Solar Ultraviolet Imager onboard Geostationary Operational Environmental Satellites ({\it GOES}/SUVI, \cite{suvi}), Extreme Ultraviolet Imager onboard Solar Orbiter ({\it SO}/EUI, \citep{eui}), the Extreme Ultraviolet Imager on Solar Terrestrial Relations Observatory-A ({\it STEREO-A}/EUVI, \cite{euvi}) in X-ray with {\it Hinode} X-ray Telescope ({\it Hindoe}/XRT, \cite{xrt}), The Reuven Ramaty High-Energy Solar Spectroscopic Imager ({\it RHESSI}, \cite{rhessi}) and the Spectrometer/Telescope for Imaging X-rays on Solar Orbiter ({\it SO}/STIX, \cite{stix}) among many others. These observations probed the coronal manifestation of the Solar flares, while the Chromospheric and Transition region counterparts of the flares were observed with The Transition Region and Coronal Explorer ({\it TRACE}, \cite{trace}), the Solar Ultraviolet Measurements of Emitted Radiation on {\it SoHO} ({\it SoHO}/SUMER, \cite{sumer}) and the Interface Region Imaging Spectrograph ({\it IRIS}, \cite{iris}). There was a lack of continuous coverage of the full solar disk in the NUV wavelength range.

The Solar Ultraviolet Imaging Telescope onboard {\it Aditya-L1} ({\it Aditya-L1}/SUIT, \cite{article,ghosh16,adityal1,suit_main}) provides a targeted probe into the Chromosphere and Transition region. It provides continuous full-disk and Region of Interest (RoI) coverage of the Sun in eleven pass bands. The details of these bands are provided in Tab.~\ref{sc_comb_fil}. The eight narrow bands provide coverage across the Mg II k and h lines, Ca II h line, the CN band, red and blue wing of the Mg II window and parts of the NUV continuum. This provides unprecedented coverage of the Chromospheric and Transition region structure of solar flares.

NOAA AR 13590 was visible on the north-east of the Solar disk on February 22, 2024. The AR was around a large sunspot in a cluster of sunspots, accompanied by a complex magnetic field structure. The active region flared multiple times during the same day, including an X1.7 flare that peaked around $\sim$ 06:32 UT, an M4.8 flare that peaked around $\sim$ 20:46 UT and an X6.3 flare that peaked $\sim$ 22:34 UT . \suit~is equipped with an onboard flare detection algorithm. Once the flare detection algorithm flags a flare and localizes the position of the flare on the detector, the program sequence prioritizes reading a fixed smaller Region of Interest (RoI) around the location for fast, higher cadence observation (for further details, please refer to \cite{flare_det}). The X6.3 flare was one of the first flares to be localized by the on-board flare detection algorithm. \suit~did not observe the X1.7 flare because, during that time, the payload was off-pointed to verify the stellar calibration program sequences. The flares were also observed by {\it SDO}/AIA, {\it SO}/STIX. {\it IRIS} observed the eastern edge of the X6.3 flare ribbons in a small [66",62"] field of view (FoV) with a 4 step raster and 15 s raster cadence.

We show the light curve arising from the whole RoI observation in comparison to {\it GOES} flux in Fig.~\ref{fig:flare_full}. In Fig.~\ref{fig:flare_full}.b, we plot the AIA 1600 {\AA} (dashed blue line) and AIA 1700 {\AA} (black dot-dashed line), in comparison to the {\it GOES} 1 {--} 8 {\AA} light curve (Solid red line) for both the flares. In Fig.~\ref{fig:flare_full}.b, we show the GOES 1 {--} 8 {\AA} light curve in comparison to the NB3 (Mg II k 279.6 nm, black dotted), NB4 (Mg II h 280.3 nm, green dot-dashed) and NB8 (Ca II h 396.85 nm, magenta dashed) light curves. NB4 light curve is offset from NB3 by -0.1 for better visibility. The NB8 light curve exhibits lower contrast and does not show a sharp peak compared to NB3 and NB4. The other interesting trend is exhibited by the continuum channels NB5, NB6 and NB7 in Fig.~\ref{fig:flare_full}.c, as a conspicuous rise in the continuum intensity is seen after both flares. In Fig.~\ref{fig:flare_full}.d we plot the {\it GOES} 1 {--} 8 {\AA} light curve in comparison to the STIX hard (25 {--} 50 keV, black dashed) and soft (5 {--} 10 keV, green dotted) X-ray light curve. The hard X-ray observation from STIX peaks at a time similar to that of NB3, NB4, and NB8.

The X6.3 flare provides a good example of the response of the local plasma environment to the flare in the Near Ultraviolet (NUV) regime in 200 {--} 400 nm. The flare peaked around $\sim$ 22:34 UT in the {\it GOES} observation. Images from six narrow band (NB) channels of \suit~ are shown in Fig.~\ref{fig:flare_nb3_peak} top panel at around $\sim$ 22:28-22:29 UT. This is at the peak of the NB3 (Mg II k 279.6 nm) channel, as observed by \suit. The 60\% peak intensity contour of the NB3 intensity is marked with the black line in all figures of Fig.~\ref{fig:flare_nb3_peak} top panel. From the figure, we also see a similar structure in the NB4 (Mg II h 280.3 nm) and NB8 (Ca II h 396.9 nm). No similar structure is observed in the other continuum channels of Fig.~\ref{fig:flare_nb3_peak} top panel.

In Fig.~\ref{fig:flare_nb3_peak} bottom panel, we show the flare observations by \suit~ at their respective peaks. Again, we see very similar structures in NB3, NB4, and NB8. NB3 and NB4 peak at almost the same time $\sim$ 22:29 UT. Although NB8 shows a similar structure, the peak intensity of NB8 is observed slightly later around $\sim$ 22:29:41 UT. The peak intensities of the other continuum NB channels are observed progressively later. The NB5 (Red wing of the Mg window) peaks around $\sim$ 22:32:41 UT. Faint signatures of flare brightening are observed in NB5, marked with red arrows in Fig.~\ref{fig:flare_nb3_peak} bottom panel. No such signatures are observed as clearly in NB6 and NB7. Both of these channels peak around $\sim$ 22:28 UT.

%----------------------------------------------------
\begin{table}
\centering
\begin{tabular}{lcr}
\hline
Filter name 	& Central 		    		 & Science\\
		& Wavelength	    	&         Target\\
		& (nm)			    	&          \\
\hline
NB1 		& 214.0                 			&Continuum\\
NB2 		& 276.7 		       			&Continuum\\
NB3 		& 279.6 			   			&Mg~\rm{II}~k\\
NB4 		& 280.3 			   			&Mg~\rm{II}~h\\
NB5 		& 283.2 			   			&Continuum\\
NB6 		& 300.0   			   			&Continuum\\
NB7 		& 388.0   			   			&CN Band\\ 
NB8 		& 396.85 			  				&Ca~\rm{II}~h\\
BB1 		& 200{--}242 		   			&Herzberg Continuum\\
BB2 		& 242{--}300 		   			&Hartley Band\\
BB3 		& 300{--}360 		   			&Huggins Band\\
\hline
\end{tabular}
\caption{Science filters, central wavelength and science target of the 11 science filters for SUIT.} 
\label{sc_comb_fil}
\end{table}
%----------------------------------------------------

In Fig.~\ref{fig:flare_lc_suit}, we plot the light curve of the event to compare the observations across various bands. The AIA and \suit~ light curves in Fig.~\ref{fig:flare_lc_suit} are calculated by adding the counts within the region of 60\% peak intensity contour of NB3 (shown in Fig.~\ref{fig:flare_nb3_peak} top panel), after co-aligning and registering the AIA and \suit~observations and normalizing them to the peak intensity. In Fig.~\ref{fig:flare_lc_suit}.a, we show the GOES 1 {--} 8 {\AA} light curve in comparison to AIA 1600 {\AA} and AIA 1700 {\AA}. The AIA 1600 and 1700 {\AA} light curve peaks around $\sim$ 5 minutes earlier than the {\it GOES} peak. 

In Fig.~\ref{fig:flare_lc_suit}.b, we show the GOES 1 {--} 8 {\AA} light curve in comparison to the NB3, NB4 and NB8 light curves. All the NB light curves behave remarkably similarly. The vertical dotted black line across all the panels in Fig.~\ref{fig:flare_lc_suit} denotes the peak intensity in NB3. NB3, NB4, and NB8 peaks around $\sim$ 22:29 UT, which is very similar to AIA 1600 {\AA} and 1700 {\AA}. We also plot the {\it GONG}-H$\alpha$ light curve from the \suit~contour region. The {\it GONG} light curve also peaks at around $\sim$ 22:29 UT. Both NB8 and {\it GONG}-H$\alpha$ shows less contrast variation in the light curve than NB3 and NB4.

We show the GOES 1 {--} 8 {\AA} light curve in comparison to the NB5 (Red wing of the Mg lines, blue dashed), NB6 and NB7 continuum channels in Fig.~\ref{fig:flare_lc_suit}.c. NB5 shows signs of flare response, although much weaker than NB3, NB4 and NB8. NB5 peaks around $\sim$ 22:31 UT, about $\sim$ 3 minutes later than NB3. Similar traits were observed from the images also, as pointed out in Fig.~\ref{fig:flare_nb3_peak} bottom panel and the accompanying discussions. More interestingly, NB6 and NB7 do not show the hallmark sign of a flare light curve, i.e. gradual increase and decrease in the intensity. These bands exhibit a slow but steady rise in intensity after the flare. Finally, in Fig.~\ref{fig:flare_lc_suit}.d, we show the {\it GOES} 1 {--} 8 {\AA} light curve in comparison to the STIX hard and soft X-ray light curve. The hard X-ray peaks around a similar time around $\sim$ 22:29 UT, similar to NB3, NB4 and AIA 1600 and 1700 {\AA}.

The Solar X-ray monitor on {\it Chandrayan-2}\citep[{\it Chandrayan-2}/XSM,][]{xsm} provides sun as a star soft X-ray spectra in 1{--}15 keV with 1s cadence and significantly better spectral resolution (180 eV at 5.9 keV) compared to STIX (1 keV ar 5.9 keV). The high spectral and temporal resolution allows us to measure the change in elemental abundances over the duration of the flares. XSM observed both flares with coverage in both impulsive and decay phase. The existing studies with XSM has successfully studies the thermal and elemental abundance evolution of various flares \citep{mondal21,kkepa23,nama23}.

Both the M and X-class flares were observed by XSM. The XSM 1{--}8 {\AA} light curves are plotted in Fig.~\ref{fig:xsm-obs} top panel. The red shaded region marks the time window where the Be filter was inserted to prevent saturation. There are periodic gaps in the data due to periodic lunar occultation. We show the spectra obtained by XSM in Fig.~\ref{fig:xsm-obs} bottom panel, during various phases of the flare. The violet spectra at 22:33 UT is near the flare peak, and shows a sharp decrease below 3 keV due to the insertion of the Be filter to avoid saturation. Various Mg, Al, S and Si lines are visible in the pre-flare spectra. In stark contrast during impulsive and peak of the flare we see strong signatures of Ca, Ar and Fe. Due to the uncertainty of the spectra during the Be window observation in $<~3~\mathrm{keV}$ regime, we fit the spectra beyond 3 keV throughput this analysis. We fit the spectra with the XSPEC model {\it chisoth}, which uses a wide range of pre-calculated spectra to fit the observed spectra (For more details please refer to the appendix of \cite{mondal21}).

One of the key observations we make is a presence of systematic residual around the Fe complex around $\sim$ 6.5 keV. This is illustrated in the top panel of Fig.~\ref{fig:xsm_fit} top panel, where there is an excess visible in the blue side of the Fe line complex at $\sim$ 6.7 keV. \cite{mithun22} could not explain this systematic excess with multithermal DEM distributions. {\bf Is it possible we are missing some line in ``chisoth"?} One of the possible explanation of this excess flux is Fe fluorescence emission. There have been observations of Fe line fluorescence at 6.4 keV (Fe K$\alpha$) and 7.06 keV (Fe K$\beta$) \citep{neupert67,doscheck71,bai79,tanaka84,parmar84,phillips12} using high-resolution X-ray spectra from Bent Crystal Spectrometer on-board Solar Maximum Mission \citep[Bent/{\it SMM},][]{bent,smm} and Yokoh \citep{yokoh} mission. Previous studies have suggested that this emission arises from the excitation of low ionization state Fe in the Photosphere either via the X-ray from the flaring plasma \citep{bai79} or directly from the non-thermal electron beam \citep{phillips73}. The Fe K$\alpha$ fluorescence is usually dependent on the position on the solar disk \citep{parmar84}. The emergent Fe K$\alpha$ emission suffers significant absorption and scattering along the line of sight, which increases with increasing heliocentric angle, resulting in a decrease in the observed intensity. For our event, the flaring region is near the disk center making the Fe K$\alpha$ fluorescence a valid candidate for explaining the excess.

We add a Gaussian line component to our fit at 6.4 keV to explore the possibility of the excess flux arising from Fe K$\alpha$ fluorescence. We find that this fits the observed spectra better than previous instances, as demonstrated in Fig.~\ref{fig:xsm_fit} bottom panel. The K$\alpha$ emission would arise from the X-ray emission from the flaring regions having energies $>$ 7.12 keV, the K edge of Fe, exciting the Fe atoms in the Photosphere. We show the intensity of the fitted Fe 6.4 keV component excess (blue solid line), in comparison to the fitted flux in 7.12 {--} 8.5 keV (green dot-dashed line) in Fig.~\ref{fig:fe_excess} top panel. For reference, {\it GOES} 1{--}8 {\AA} soft X-ray flux (red dotted line) and STIX 25 {--} 50 keV Hard X-ray flux (black dashed line) are overplotted. STIX Hard X-ray is a fair representative of the non-thermal electron flux deposited into the foot points. In the bottom panel, the light curve fitted Fe 6.4 keV excess Gaussian component is plotted with the lightcurve from the bright kernels marked with the two boxes in Fig.~\ref{fig:flare_nb3_peak} bottom panel. In both panels, the peak time of the Fe excess (blue solid line), STIX hard X-ray (black dashed line) and NB5 brightness from box 1 (dotted magenta line) are marked with vertical lines.

This article reports the \suit~narrowband imaging of the first localized flare by the onboard flare detection algorithm. We report the observation in NB3 (Mg II k 279.6 nm), NB4 (Mg II h 280.3 nm), NB8 (Ca II h 396.9 nm) and the continuum channels NB5 (Red wing of Mg II), NB6 and NB7. For both the flares, the NB3, NB4 and NB8 peak around the same time as AIA 1600 and 1700 {\AA}. For the X6.3 flare, NB3, NB4 behave very similarly to NB8 within the NB3 intensity contour (see Fig.~\ref{fig:flare_lc_suit}.b). But the NB8 peak over the whole active region is much broader compared to the NB3 and NB4 light curves (see Fig.~\ref{fig:flare_full}). This implies that NB8 behaves differently in other parts of the active region.

The NB5 observes the continuum that is usually attributed as the Balmer continuum. There have been previous studies where Photospheric metal lines went into emission and affected the Balmer continuum \citep{heinzel14,kleint17}. The dominant contribution would still be the Balmer continuum. \cite{reetika21} attributed the brightening in the SJI 2832 {\AA} continuum for a mini flare to direct signature of electron beams. \cite{kowalski19} showed for one event, that the SJI 2832 {\AA} continuum enhancement and several Phototspheric absorption lines going into emission can be attributed to significant Photospheric heating. The entire 2832 {\AA} window of IRIS had several \ion{Fe}{2} and \ion{Cr}{2} lines which are usually observed as absorption lines, in emission. Curiously enough, this observation was also made in a bright umbral flare kernel, similar to the current event. As there was no {\it IRIS} scan of the umbral brightening visible in NB5, unfortunately, we can not comment on the spectral nature of the bright kernel.

We see an excess around the 6.5 keV Fe complex from the XSM observations which can be fitted with a single Gaussian. From Fig.~\ref{fig:fe_excess}, the Fe excess light curve (blue solid line) behaves very similar to the soft X-ray flux beyond the Fe K edge at 7.12 keV (green dot-dashed line). The excess also shows no correlation with the STIX 25 {--} 50 keV hard X-ray flux (black dashed line), illustrating no significant contribution from the non-thermal electron flux. This suggests that the excess flux seen around 6.4 keV Fe complex arises from the Fe fluorescence from the flaring X-ray. If we assume the penumbral brightening observed in this flare to be similar as observed by \cite{kowalski19}, the bright kernels observed in NB5 mainly arises from a plethora of \ion{Fe}{2} lines. The flaring X-ray beyond Fe K edge (7.12 keV) photoionizes the Fe in the Photosphere, giving rise to both \ion{Fe}{2} lines in the red wing of Mg II, along with the observed Fe fluorescence by XSM. 

We see a similar brightening in NB2, the blue wing of the Mg window. The light curve of the brightening of the blue wing of Mg is shown in Fig.~\ref{fig:fe_excess} third panel. The light curve shows peak at both hard X-ray peak and later on closer to Fe excess component peak. This possibly shows both Photospheric and Chromospheric components from the NB2. Further investigation and modelling is required to comment on the local plasma parameters that would produce the bright kernels observed in both NB5 and NB2, with the relative timing within themselves and also in comparison to the various energies in X-ray.

The other interesting observation is the rise in the continuum intensity, specifically in NB6 and NB7 as the flares happen. We can see a steady rise in the continuum intensity after the M and X class flare (see Fig.~\ref{fig:flare_full}). For the X6.3 flare, we do see some signature of the flare in NB5, although it peaks about $\sim$ 5 minutes later compared to NB3 and NB4. The photospheric nature of the NB5 continuum explains the 5-minute delay of the peak from NB3. We see the flare peak in sequential order of formation height NB3 and NB4 (Mg II, 22:29 UT) and NB8 (Ca II) $\Longrightarrow$ NB5 (Photospheric continuum, 22:32:41 UT). We are observing the increase in continuum intensity from 283.2 nm to 388 nm. The consistent increase in continuum intensity is happening across the Balmer jump ($\lambda$~=~364.5 nm).

% Your references go at the end of the main text, and before the
% figures.  For this document we've used BibTeX, the .bib file
% scibib.bib, and the .bst file Science.bst.  The package scicite.sty
% was included to format the reference numbers according to *Science*
% style.


\bibliography{mybib}

\bibliographystyle{Science}



% Following is a new environment, {scilastnote}, that's defined in the
% preamble and that allows authors to add a reference at the end of the
% list that's not signaled in the text; such references are used in
% *Science* for acknowledgments of funding, help, etc.

\begin{scilastnote}
\item We've included in the template file \texttt{scifile.tex} a new
environment, \texttt{\{scilastnote\}}, that generates a numbered final
citation without a corresponding signal in the text.  This environment
can be used to generate a final numbered reference containing
acknowledgments, sources of funding, and the like, per {\it Science\/}
style.  Along those lines, we'd like to thank readers of this document
for their attention, and invite them to address any questions to
Stewart Wills, at swills@aaas.org.
\end{scilastnote}

%%%%%%%%%
\begin{figure}
    \centering
    \includegraphics[width=0.8\textwidth,trim={2.3cm 2.5cm 1cm 4.5cm},clip]{lc_full_suit_contour.pdf}
    \caption{Light curves from the whole RoI FoV of \suit~observations compared with AIA, {\it GOES} and STIX observation for the two flares.}
    \label{fig:flare_full}
\end{figure}
%%%%%%%%%

%%%%%%%%%
\begin{figure}
    \centering
    \includegraphics[trim={0cm 0.65cm 0cm 0cm},clip,width=0.5\textwidth]{suit_roi_nb3_peak.pdf} \\
    \includegraphics[width=0.5\textwidth]{suit_roi_all_peak.pdf}
    \caption{\suit~observations of the flare from various NB filters. Top panel: during NB3 peak. Bottom panel: During the peak of individual bands.}
    \label{fig:flare_nb3_peak}
\end{figure}
%%%%%%%%%

%%%%%%%%%
\begin{figure}
    \centering
    \includegraphics[width=0.8\textwidth,trim={2.3cm 2.5cm 0cm 4.5cm},clip]{lc_suit_contour.pdf}
    \caption{Light curves from the pixels within intensity contour picked from \suit~and co-aligned AIA observations. The NB4 light curve in the second panel is offset by -0.1 from NB3 for better visibility. The vertical dotted dark line marks the flare's peak in NB3 observation.}
    \label{fig:flare_lc_suit}
\end{figure}
%%%%%%%%%

%%%%%%%%
\begin{figure}
\centering
    \includegraphics[trim={0.5cm 3.3cm 0.8cm 4cm}, clip, width=0.8\textwidth]{xsm_lc.pdf} \\
    \includegraphics[trim={1.8cm 2.5cm 3cm 3cm},clip,width=0.75\textwidth]{xsm_spec.pdf}
    \caption{Top panel: XSM observation of the two flares. The red shaded region marks the window when the Be filter was inserted. Bottom panel: X-ray spectra during various phases of the flare. The violet spectra is during the soft X-ray peak of the flare and shows a sharp decrease in the intensity below 3 keV. This is due to the insertion of Be filter to avoid saturation.}
    \label{fig:xsm-obs}
\end{figure}
%%%%%%%%

%%%%%%%%%%
\begin{figure}
\centering
    \includegraphics[trim={0.5cm 1cm 0.5cm 0.7cm}, clip, width=0.8\textwidth]{xsm_fit.pdf}
    \caption{XSM spectra in 3 {--} 8.5 keV binned between 22:27:20 {--} 22:27:25 UT. Top panel shows the fit with ``chisoth+chisoth" model. Bottom panel shows the same spectra fitted with ``chisoth+chisoth+gaussian" model.}
    \label{fig:xsm_fit}
\end{figure}
%%%%%%%%%%

%%%%%%%%%%
\begin{figure*}
\centering
    \includegraphics[trim={1.3cm 0.1cm 0.3cm 1.2cm}, clip, width=0.8\textwidth]{fe_excess_4.pdf}
    \caption{Fe excess emission from 6.4 keV(blue solid) in comparison to STIX 25-50 keV(black dashed), GOES 1 {--} 8 $\AA$ (dotted red) and XSM 7.12 {--} 8.5 keV (green dot-dashed) light curve.}
    \label{fig:fe_excess}
\end{figure*}
%%%%%%%%%%


% For your review copy (i.e., the file you initially send in for
% evaluation), you can use the {figure} environment and the
% \includegraphics command to stream your figures into the text, placing
% all figures at the end.  For the final, revised manuscript for
% acceptance and production, however, PostScript or other graphics
% should not be streamed into your compliled file.  Instead, set
% captions as simple paragraphs (with a \noindent tag), setting them
% off from the rest of the text with a \clearpage as shown  below, and
% submit figures as separate files according to the Art Department's
% instructions.


\clearpage

\noindent {\bf Fig. 1.} Please do not use figure environments to set
up your figures in the final (post-peer-review) draft, do not include graphics in your
source code, and do not cite figures in the text using \LaTeX\
\verb+\ref+ commands.  Instead, simply refer to the figure numbers in
the text per {\it Science\/} style, and include the list of captions at
the end of the document, coded as ordinary paragraphs as shown in the
\texttt{scifile.tex} template file.  Your actual figure files should
be submitted separately.

\bibliography{mybib}{}
\bibliographystyle{aasjournal}

\end{document}




















